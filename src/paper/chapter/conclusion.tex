%summary: tell it a bit differently.
% limitations: external validity: Privileged sample, no causality (in the logits), internal validity: didn't benchmark against another method, measurement: only survey-reported
% implications for future research: Parenting Styles models should be extended using the number of children as an explicit input to the parenting styles function, want to test using only parents of first-borns.
This paper uses detailed data on the early childhood environment and unsupervised machine learning to produce a classification of parenting styles and to assess its properties. I show that the classification's properties are only partly in line with the theoretical predictions put forward in \textcites{doepkeParentingStyleAltruism2017}{doepkeEconomicsParenting2019}. Furthermore, I demonstrate that part of the variation in parenting styles is explained by how much time parents invest in their children and that time invested is negatively correlated with the number of children in the family. 

My paper is limited by three factors. First, my results cannot speak to the question which factors \textit{causally} determine the choice of parenting styles. For example, when preferences for a permissive parenting style, that is, low levels of paternalism, are correlated with preferences about the number of children, then it is not necessarily true that more children make parents more permissive. Instead, they could have been more permissive in the first place and took fertility (and marriage) decisions accordingly. One way to circumvent this problem is to only look at parents who currently have one child and investigate whether the variation in parenting styles is the same as for parents irrespective of the number of children.

Second, the classification's robustness could be enhanced by applying other methods to this problem and see whether similar conclusions can be drawn from their results. As alluded to previously, clustering algorithms such as $K$-means or Hierarchical Clustering could easily be implemented and benchmarked against the results from the GMM presented in this paper.    

Third, the data used in this paper suffer from non-random attrition in the sense that low-socio-economic status families drop out more frequently than their high-SES counterparts. If the choice of parenting style is correlated with SESs as the summary statistics in Table \ref{tab:summ_stats} suggest, the non-random missingness introduces bias to my results.

Assuming that my results are at least partially driven by the differences in time investments due to differences in the number of children, future research on parenting styles needs to account for fertility choices. As already noted by \textcite{doepkeEconomicsParenting2019}, fertility and parenting decisions are closely linked. My results are suggestive that omitting fertility might lead to spurious conclusions when studying the choice of parenting styles empirically.