Data used in this study is drawn from the newborn cohort of the German National Educational Panel Study \parencite[NEPS][]{nepsnationaleducationalpanelstudybamberggermanyNEPSStartingCohort2021}. The newborn cohort is a representative longitudinal dataset on 3481 children born between February and June 2012 in Germany. The survey instruments focus on the children's education and competence development and are conducted with both children and parents as target persons. Children are 6, 16, 25, 37 months old in the first four waves of the survey, respectively. Afterwards, parents and children are interviewed once a year up until the most recent wave where children are seven years old. 

I exploit information on parenting styles and measures of the socio-economic status of the household reported by parents and measures of parent-child interaction behaviors rated by interviewers. I collapse the panel dataset to a cross-section for two reasons. First, many measures of interest are elicited only once (measures of parenting styles and most variables on the socio-economic environment). If they are elicited repeatedly, I average the responses at the individual level to reduce measurement error. Second, the algorithm I use to classify parenting styles assumes independent and identically distributed observations, an assumption which is not defendable for panel data.
% many parental and child behaviors are age-dependent and are not easily comparable across waves.

\textit{Parenting Styles}: Measures of various parenting practices are elicited from parents in waves five to eight. Parents are asked how often certain things occur between them and their child. Answers are given on five-point Likert scales ranging from 1-``never'' to 5-``very often''. The dimensions elicited read powerful enforcement, emotional warmth, inconsistent parenting, negative communication, monitoring, autonomy, positive parenting behavior, and psychological control. Given that parenting practices are related to the child's age, I de-mean the scores of the items using wave-specific means if measures are elicited repeatedly. Each dimension consists of three to four sub-items whose scores I average at the individual level. I drop observations for whom at least one of the dimensions is not available. This leaves me with 1504 observations whose scores are directly fed into the classification algorithm explained in \ref{sec:classifying}.

\textit{Parental interaction behaviors}: To measure the quality of the typical parent-child interaction in each household, one of the parents and the child are asked to play together for five minutes during the interview in waves 1 to 3 \parencite[for a more detailed description, see]{linbergQualityParentChildInteractions2019}. The interactions are videotaped and rated afterwards by trained professionals along various dimensions. The behaviors should capture to what extent the child is supported emotionally and whether or not its development is stimulated in an appropriate manner. They are therefore well-suited to enrich the analyses of parenting styles. Item scores range from 1-``not at all characteristic'' to 5-``highly characteristic''. The first dimension is sensitivity to distress and ought to capture how sensitive the parent reacts to the child's gestures and expressions. Intrusiveness measures the extent to which the parent actively limits the child's autonomy, whereas detachment describes to what degree parents are missing the child's signals. Finally, the stimulation dimension assesses whether the parent fosters the child's cognitive development while emotionality evaluates whether the parent displays appropriate emotions frequently.

Table \ref{tab:summ_stats} show summary statistics stratified by the household's socio-economic status (SES). I consider households to be low-SES if their number of books at home is below 100.\footnote{
This measure for the socio-economic environment of children is used quite frequently in the literature. See, e.g., \textcite{resnjanskijCanMentoringAlleviate} for a recent example.
}
%
\begin{table}[!htbp]
	\centering
	\begin{threeparttable}
		\caption{Summary Statistics}
		\label{tab:summ_stats}
		\begin{tabular}[t]{lcccccc}
			\hline\hline\\[-1.8ex] 
			\multicolumn{1}{c}{ } & \multicolumn{2}{c}{High (N=1032)} & \multicolumn{2}{c}{Low (N=395)} & \multicolumn{1}{c}{    } & \multicolumn{1}{c}{    } \\
			\cmidrule(l{3pt}r{3pt}){2-3} \cmidrule(l{3pt}r{3pt}){4-5}
			& Mean & Std. Dev. & Mean  & Std. Dev.  & Mean Diff. & p-val.\\
			\midrule
			A. Child characteristics &&&&&&           \\
			\hspace{5mm}Female & 0.495 & 0.500 & 0.491 & 0.501 & -0.004 & 0.892\\
			\hspace{5mm}Migration back. & 0.316 & 0.463 & 0.462 & 0.499 & 0.146 & 0.000\\
			\hspace{5mm}Siblings & 0.736 & 0.871 & 0.653 & 0.903 & -0.083 & 0.116\\
			B. Family characteristics &&&&&&           \\
			\hspace{5mm}Married & 0.803 & 0.349 & 0.705 & 0.404 & -0.099 & 0.000\\
			\hspace{5mm}High School & 0.784 & 0.412 & 0.333 & 0.472 & -0.450 & 0.000\\
			\hspace{5mm}Unemployed & 0.199 & 0.398 & 0.233 & 0.423 & 0.034 & 0.174\\
			\hspace{5mm}Mother's age & 33.663 & 4.271 & 31.312 & 5.022 & -2.350 & 0.000\\
			\hspace{5mm}Income & 4496.568 & 2274.691 & 3308.075 & 1294.929 & -1188.494 & 0.000\\
			\hspace{5mm}Time investment* & 0.080 & 0.978 & -0.236 & 1.037 & -0.316 & 0.000\\
			C. Interaction behaviors &&&&&&           \\
			\hspace{5mm}Sensitivity* & 0.108 & 0.939 & -0.316 & 1.076 & -0.424 & 0.000\\
			\hspace{5mm}Intrusiveness* & -0.074 & 0.937 & 0.257 & 1.143 & 0.331 & 0.000\\
			\hspace{5mm}Detachment* & -0.023 & 0.949 & 0.069 & 1.114 & 0.092 & 0.185\\
			\hspace{5mm}Stimulation* & 0.072 & 0.967 & -0.185 & 1.058 & -0.258 & 0.000\\
			%\hspace{5mm}Pos. Regard* & 0.082 & 0.967 & -0.212 & 1.043 & -0.294 & 0.000\\
			%\hspace{5mm}Neg. Regard* & -0.048 & 0.927 & 0.179 & 1.212 & 0.227 & 0.002\\
			\hspace{5mm}Emotionality* & 0.098 & 0.961 & -0.194 & 1.044 & -0.292 & 0.000\\
			\hline\hline
		\end{tabular}
	\begin{tablenotes}
		\small
		\item \textit{Notes}: Summary statistics of $N = 1427$ observations stratified by SES. P-values for $t$-tests on equality of means against a two-sided alternative. Female is a dummy indicating whether a child's sex is female. Migration background indicates whether the child has any migration background. Siblings is a dummy indicating whether there are siblings born prior to the surveyed child's birth. Married is a dummy indicating whether the respondent is married. Unemployed is a dummy indicating whether the respondent was unemployed one year prior to child birth. Time investment is generated as the sum of a range of items capturing the frequency of highly-interactive parent-child activities. *: Variable has been standardized to have mean zero and standard deviation equal to one.
	\end{tablenotes}
	\end{threeparttable}
\end{table}
%
The low number low-SES families in my core sample is driven by SES-selected attrition in the sense that low-SES families drop out of the sample more frequently than high-SES families. Children in high-SES households are less often from a migrant background and have more siblings when born. Furthermore, parents in high-SES households are more often married, better educated, less often unemployed prior to birth, older at birth, richer, and engage more frequently in highly interactive parent-child activities. Regarding parental interaction behaviors during the play scene, high-SES parents are more sensitive, less intrusive, less detached, more stimulating, and display appropriate emotions more frequently.