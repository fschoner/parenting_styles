I use the self-reported items (explained in sec 2) capturing different dimensions of parenting behavior to come up with a classification of different parenting styles in an unsupervised manner using Gaussian mixture modelling.
%
\subsection{Gaussian Mixture Model}
I model the scores derived from self-reported survey items on parenting styles as a Gaussian Mixture Model. Strictly speaking, the Gaussian mixture model (GMM) is a density estimation technique. At the same time, however, it is a clustering algorithm that partitions observations into classes (``clusters'') and provides probabilistic class assignments for each observation. Each class is represented by an own density.\footnote{
	This section borrows heavily from \textcite{hastieElementsStatisticalLearning2009}, chapters 6, 8, and 14.
} 

Let $x \in \mathbb{R}^p$ denote the vector of scores on the parenting styles items and denote by $f: \mathbb{R}^p \longrightarrow \mathbb{R}_{+}$ its density.\footnote{
	$p=8$.
} Let $i = 1,\ldots,N$ denote the observations. The GMM postulates that 
\begin{equation}
	f(x) = \sum_{m=1}^{M} \alpha_m \phi(x; \mu_m, \bm{\Sigma}_m),
\end{equation}
where $\alpha_m$ are weights (or mixing proportions) that sum to 1, and $\phi$ ist the multivariate normal density with mean vector $\mu$ and variance-covariance matrix $\bm{\Sigma}$. In other words, the density of $x$ is assumed to be represented by a mixture of densities which are allowed to have different weights. Mixtures tend to outperform off-the-shelf density estimation if greater flexibility is needed, such as when the data is multimodal.
Furthermore, and more importantly for my application, the GMM provides estimates of the probability that an observations belongs to class $m$. Observation $i$ is then assigned to the class with the highest probability, denoted by $\widehat{\delta_i}$. To see this, note that the estimation results contain $\widehat{\mu}_m$, $\widehat{\bm{\Sigma}}_m$, and $\widehat{\alpha}_m$ for classes $m = 1,\ldots, M$. Then,
\begin{equation*}
	\widehat{\delta}_i = \max_{1 \leqslant m \leqslant M} \widehat{\text{Pr}}(i \in m) = \frac{\widehat{\alpha}_m \phi(x_i; \widehat{\mu}_m, \widehat{\bm{\Sigma}}_m)}{\sum_{k=1}^{M} \widehat{\alpha}_k \phi(x_i; \widehat{\mu}_k, \widehat{\bm{\Sigma}}_k)}
\end{equation*}
Estimation is conducted using the Expectation-Maximization algorithm, a variant of maximum likelihood estimation usually applied when direct maximization of a likelihood is numerically difficult. For each number of classes, the Bayesian Information Criterion (BIC) is computed and the maximizing number is picked.

Note that many other methods could be used to detect latent concepts in the data. For example, principal components analysis is very popular in the economic literature on parenting styles (e.g. \cite{ermischOriginsSocialImmobility2008, fioriniHowAllocationChildren2014, bonoEarlyMaternalTime2016, cobb-clarkParentingStyleInvestment2019, zumbuehlParentalInvolvementIntergenerational2020}). However, while these methods are able to describe latent factors in the data by using linear combinations of the input variables, they do not come with a natural classification procedure which is essential for the aim of this paper. Furthermore, other popular clustering algorithms that provide class assignments, in particular $K$-means, could have been used. One major advantage of the GMM over such \textit{deterministic} algorithms is that the GMM provides probabilities for all classes, enabling the researcher to assess the robustness of results based on the classification by, say, discarding observations with uncertain assignments.
%
%
\subsection{Results}
Table \ref{tab:class} displays means of the scores of parenting style dimensions by class and $p$-values for the equality of the respective class-means. The BIC is maximized for three classes. According to the classification, 42\%, 16\%, and 41\% of the observations belong to classes 1, 2, and 3, respectively.

My subjective classification of parenting styles is mainly based on the dimensions ``powerful enforcement'', ``monitoring'', and, to a lesser extent, ``autonomy''. The remaining categories, while informative for a broader characterization of styles, do not speak directly to aspects of the model put forward in \textcite{doepkeParentingStyleAltruism2017}. First, the dimension of powerful enforcement directly relates to how often parents actively restrict their child's choice set given that they elicit how often parents set clear limits and exercise their authority. Therefore, parent with high scores along this dimension are more likely to be authoritarian. Second, authoritarian strategies that involve forcing children to obey often involve close monitoring of the offsprings' actions. The monitoring dimension captures to what extent parents know the whereabouts of their children and what activity they pursue. Therefore, an authoritarian parenting style should come with a high score on the monitoring dimension. Conversely, permissive parents refrain from influencing a child's choice, that is, they monitor less and enforce their will much more seldomly than authoritarian parents and are therefore characterized by lower scores along these dimensions. Finally, the autonomy dimension captures whether the child is asked for its opinion and allowed to express it and whether it is explained the parental demands on it. These aspects relate to both authoritative and permissive parenting styles: One the one hand, to mold children's preferences, parents need t reason why they endorse or disapprove of certain activities of the child. On the other hand, permissive parents value th independence of their child and should therefore be inclined to to hear their child's opinion. Thus, authoritative and permissive parents should score higher on the autonomy dimension.
%
\begin{table}[!htbp]
	\centering
	\begin{threeparttable}
		\caption{Classification Results}
		\label{tab:class}
		\begin{tabular}{lcccccc}
			\hline \hline\\[-1.8ex] 
			&    &    &    &\multicolumn{3}{c}{p-values} \\ 
			\cline{5-7} \\[-1.8ex]Dimension & 1 & 2 & 3 & 1/2 & 1/3 & 2/3 \\ 
			\midrule
			Powerful enforcement & 0.033 & 0.057 & -0.066 & 0.596 & 0.001 & 0.004 \\ 
			Emotional warmth & 0.130 & 0.370 & -0.241 & 0.000 & 0.000 & 0.000 \\ 
			Inconsistent parent. & -0.101 & -0.095 & 0.112 & 0.641 & 0.000 & 0.000 \\ 
			Neg. communication & -0.076 & -0.235 & 0.177 & 0.000 & 0.000 & 0.000 \\ 
			Monitoring & 0.099 & 0.290 & -0.199 & 0.000 & 0.000 & 0.000 \\ 
			Autonomy & -0.007 & 0.191 & -0.066 & 0.000 & 0.000 & 0.000 \\ 
			Pos. parent. behavior & -0.097 & 0.669 & -0.136 & 0.000 & 0.485 & 0.000 \\ 
			Psychological control & -0.127 & -0.100 & 0.137 & 0.178 & 0.000 & 0.000 \\ 
			\hline \bottomrule
		\end{tabular}
		\begin{tablenotes}
			\small
			\item \textit{Notes}: Class means of the different parenting styles dimensions based on $N = 1504$ observations. P-values for $t$-tests on equality of means against a two-sided alternative.
		\end{tablenotes}
	\end{threeparttable}
\end{table}
%
Table \ref{tab:class} illustrates that based on the powerful enforcement and monitoring dimensions, class 2 stands out by scoring higher on the enforcement dimension than classes 1 (albeit insignificantly so) and 3, respectively. Furthermore, class-2 parents are monitoring their children more closely than those in the other classes. However, parents in class 2 also grant more autonomy to their children, a behavior we would have expected to see most among permissive parents and, to a lesser extent, among authoritative parents. Nonetheless, a preliminary classification based on the class-means suggests that authoritative parents belong to class 1 while classes 2 and 3 contain authoritarian and permissive parents, respectively. 

The remaining dimensions do not contribute to making the classification more plausible by contradicting each other. Class-3 parents are more inconsistent, that is, among other things, they find it harder to be resolute than parents in classes 1 and 3, which bolsters the claim that they adopted a permissive parenting since they probably interefere less with their children's choices. On the other hand, class-2 parents score higher on the emotional warmth dimension and exhibit more positive parenting behavior as well as score lower on the negative communication dimension, meaning that, broadly speaking, they praise and comfort their children rather than criticizing and insulting them. This contradicts the classification insofar as one would expect that adopting an authoritarian strategy goes hand in hand with a harsher tone. Given that these dimensions are only partly related to the characteristics of the different parenting styles outlines in \textcite{doepkeEconomicsParenting2019}, I stick to the classification provided above.